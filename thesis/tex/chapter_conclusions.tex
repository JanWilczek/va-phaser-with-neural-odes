%%%%%%%%%%%%%%%%%%%%%%%%%%%%%%%%%%%%%%%%%%%%%%%%%%%%%%%%%%%%%%%%%%%%%%%%%%%%%%%%%%%%%%%
\chapter{Summary and Conclusions}
\label{chapter:conclusions}
%%%%%%%%%%%%%%%%%%%%%%%%%%%%%%%%%%%%%%%%%%%%%%%%%%%%%%%%%%%%%%%%%%%%%%%%%%%%%%%%%%%%%%%

%% Presentation conclusions:
% ODENet architecture can learn the ODE governing an analog audio effect.
% ODENet cannot produce aliasing unless it is present in the training data; no need for oversampling.
% The learned derivative can be used at sampling rates higher than the training sampling rate.
% ODENet can provide results comparable to traditional recurrent neural networks (like LSTM) yet using a smaller number of parameters.
% Solver choice is not that important; derivative network treatment is.

The aim of this work was to investigate whether the ODENet framework--teaching a neural network the \ac{ODE} that governs a system and then supplying it to a numerical solver of choice--is can be used for \ac{VA} modeling. If the answer was positive, possible advantages and disadvantages in comparison to other modeling methods were to be considered.

Ultimately, the answer to this question turned out to be only partially positive. It has been proven on the example of the diode clipper that the ODENet architecture can learn the \ac{ODE} governing an analog 

% TODO: Address all open questions

%%%%%%%%%%%%%%%%%%%%%%%%%%%%%%%%%%%%%%%%%%%%%%%%%%%%%%%%%%%%%%%%%%%%%%%%%%%%%%%%%%%%%%%
\section{Future Work}
%%%%%%%%%%%%%%%%%%%%%%%%%%%%%%%%%%%%%%%%%%%%%%%%%%%%%%%%%%%%%%%%%%%%%%%%%%%%%%%%%%%%%%%

% - ODENet with white-box models
% - ODENet with STN's measurements
% - ODENet in a latent space
