%%%%%%%%%%%%%%%%%%%%%%%%%%%%%%%%%%%%%%%%%%%%%%%%%%%%%%%%%%%%%%%%%%%%%%%%%%%%%%%%%%%%%%%
\cleardoublepage
\thispagestyle{empty}
\begin{center}
\vspace*{3cm}
{\huge \bf Part I}\\ \vspace*{1cm}
{\Huge \bf Title of this Part}\\\vspace*{0.2cm}
{\Huge \bf Being Longer than One Line}\\\vspace*{3cm}
\begin{figure}[ht]
\centering
\includegraphics[height=6cm]{figures/part1_notesAndWaveform_orange}
\end{figure}
\end{center}
\addcontentsline{toc}{part}{I\hspace {1em}Title of this Part Being Longer than One Line}
\label{par:part1}
\newpage
\quad
\thispagestyle{empty}
\newpage

%%%%%%%%%%%%%%%%%%%%%%%%%%%%%%%%%%%%%%%%%%%%%%%%%%%%%%%%%%%%%%%%%%%%%%%%%%%%%%%%%%%%%%%


%%%%%%%%%%%%%%%%%%%%%%%%%%%%%%%%%%%%%%%%%%%%%%%%%%%%%%%%%%%%%%%%%%%%%%%%%%%%%%%%%%%%%%%
\chapter{Theoretic Foundations}
\label{chapter:Foundations}
%%%%%%%%%%%%%%%%%%%%%%%%%%%%%%%%%%%%%%%%%%%%%%%%%%%%%%%%%%%%%%%%%%%%%%%%%%%%%%%%%%%%%%%

Each chapter should start with a small summary discussing the content and the relation
of the sections. In this chapter, we elaborate on the theoretical background, 
foundations and concepts that are being used in the thesis. In particular, we 
focus on latex construct that are typically used in thesis documents.

When using labels, please avoid collisions in the label names.


%%%%%%%%%%%%%%%%%%%%%%%%%%%%%%%%%%%%%%%%%%%%%%%%%%%%%%%%%%%%%%%%%%%%%%%%%%%%%%%%%%%%%%%
\section{Ordinary Differential Equations}
\label{section:ode}
%%%%%%%%%%%%%%%%%%%%%%%%%%%%%%%%%%%%%%%%%%%%%%%%%%%%%%%%%%%%%%%%%%%%%%%%%%%%%%%%%%%%%%%

An \ac{ODE} is an equation of the form
\begin{equation}
  \frac{\mathrm{d} y}{\mathrm{d} t} = f(t, y),
  \label{eq:general_ode}
\end{equation}
where $y$ is the \emph{unknown function}, $t$ is the \emph{independent variable}, and $f(t, y)$ is a function of both, $t$ and $y$. Implicitly, $y$ is dependent on $t$, i.\ e., $y = y(t)$. In the scope of this work, $t$ denotes time. The term \emph{ordinary} means that $y$ is a function of a single variable and, thus, an "ordinary" derivative is used, i.\ e., $\frac{\mathrm{d} y}{\mathrm{d} t}$ \cite{Gockenbach2011}.

To \emph{solve} an \ac{ODE} means to find a function $y$ that satisfies \Equation{eq:general_ode}. Such $y$ is called a \emph{solution}.

\acp{ODE} are used to model dynamical systems \cite{Karlsson2019}. An example of a dynamical system that can be described by an \ac{ODE} is an electrical circuit containing capacitive elements, such as a diode clipper circuit \cite{Yeh2007}. % TODO: Reference a section on the diode clipper
Since an equation of the form \Equation{eq:general_ode} describes a rate of change $\frac{\mathrm{d} y}{\mathrm{d} t}$, its solution will not be a single function, but rather a family of functions or a parametrized function. To obtain a unique solution, we need to specify an \emph{\ac{IC}}. An \ac{ODE} together with an initial condition makes up an \emph{\ac{IVP}}.

Some classes of \acp{IVP} can be solved analytically. However, for the most interesting applications, e.\ g., in the domain of analog audio effects, the corresponding \acp{IVP} are solved using numerical methods, i.\ e., specialized algorithms using discrete points to approximate the true solution \cite{Gockenbach2011}. 

A group of numerical methods is called \emph{time-stepping methods}. Given a grid of time instants $t_0 < t_1 < \dots < t_n$ and a value of $y$ at $t_0$, $y(t_0$) these methods use the following identity
\begin{equation}
  y(t_{i+1}) = y(t_{i}) + \int \limits_{t_i}^{t_{i+1}} f(s, y(s)) \mathrm{d} s
  \label{eq:time_stepping_identity}
\end{equation}
to approximate the value $y$ at the points on the grid, $y(t_1), y(t_2), \dots, y(t_n)$. The core idea is to approximate the integral in \Equation{eq:time_stepping_identity}. Thus, the process of solving an \ac{IVP} is often referred to as \emph{integrating} the \ac{ODE} \cite{Gockenbach2011}.

%%%%%%%%%%%%%%%%%%%%%%%%%%%%%%%%%%%%%%%%%%%%%%%%%%%%%%%%%%%%%%%%%%%%%%%%%%%%%%%%%%%%%%%
\section{Deep Learning}
\label{section:deep_learning}
%%%%%%%%%%%%%%%%%%%%%%%%%%%%%%%%%%%%%%%%%%%%%%%%%%%%%%%%%%%%%%%%%%%%%%%%%%%%%%%%%%%%%%%

Broadly speaking, machine learning is a field of engineering using data to improve on a task at hand \cite{Goodfellow-et-al-2016}. A subfield of machine learning called deep learning uses \acp{ANN} for the improvement process. 

An \ac{ANN} is a model whose goal is to approximate some function $f^*$ \cite{Goodfellow-et-al-2016}. A \emph{feedforward network} or an \ac{MLP} is an input-to-output mapping. For example, in the context of audio signal processing, $f^*$ could be a mapping of an input signal $s_{in}(t)$ to an output signal $s_{out}(t)$, i.\ e., $s_{out}(t) = f*(s_{in}(t))$. The mapping $f$ defined by an \ac{MLP}, $\hat{s}_{out}(t) = f(s_{in}(t); \pmb{\theta})$, is dependent on the parameter vector $\pmb{\theta}$. The learning process aims at finding a parameter vector $\pmb{\theta^*}$ that allows for a sufficiently accurate approximation of $f^*$ by $f$.

% TODO: Add figure with a MLP

A feedforward network consists of \emph{layers} which define the subsequent stages of computations. The first layer is called the \emph{input layer} and the last layer is called the \emph{output layer}. The layers in between are called the \emph{hidden layers}, because we do not observe their output directly. Each layer consists of a set of \emph{units}: affine transformations of unit's inputs followed by a nonlinearity. Each unit in a layer receives outputs of all units in the previous layer at its input and outputs a single scalar. The number of layers determines the \emph{depth} of the model. Feedforward networks with at least 1 hidden layer are called \emph{deep}. The number of units in a layer determines its \emph{width}.

\acp{MLP} are \emph{universal function approximators}, i.\ e., they can approximate an arbitrary function (fulfilling very mild criteria) with an arbitrarily small error given enough width or depth \cite{Goodfellow-et-al-2016}.


%%%%%%%%%%%%%%%%%%%%%%%%%%%%%%%%%%%%%%%%%%%%%%%%%%%%%%%%%%%%%%%%%%%%%%%%%%%%%%%%%%%%%%%
\section*{Architectures}
%%%%%%%%%%%%%%%%%%%%%%%%%%%%%%%%%%%%%%%%%%%%%%%%%%%%%%%%%%%%%%%%%%%%%%%%%%%%%%%%%%%%%%%

As deep learning is a widely and actively pursued field of science, there exist many different neural network architectures beyond the \ac{MLP}. They are often designed for the task at hand using expert knowledge or heuristics \cite{Goodfellow-et-al-2016}. In this section, two highly successful architectures are presented, both of them relevant for this thesis.

%%%%%%%%%%%%%%%%%%%%%%%%%%%%%%%%%%%%%%%%%%%%%%%%%%%%%%%%%%%%%%%%%%%%%%%%%%%%%%%%%%%%%%%
\subsection*{Long Short-Term Memory}
%%%%%%%%%%%%%%%%%%%%%%%%%%%%%%%%%%%%%%%%%%%%%%%%%%%%%%%%%%%%%%%%%%%%%%%%%%%%%%%%%%%%%%%

\Ac{RNN} is a neural network that maintains a \emph{feedback} connection, i.\ e., a connection from its output to its input. Typically, to calculate the current output, the previous output is taken as part of the input. This step-by-step processing is characteristic to sequence modeling, where \acp{RNN} are found most useful.

% TODO: Add a figure with an LSTM

A particularly successful \ac{RNN} architecture is the \ac{LSTM} \cite{Hochreiter1997}. \ac{LSTM} uses \emph{gating units} to allow or disallow its \emph{memory cells} to be updated during processing. This helps to learn long-term dependencies which are crucial for the network to maintain a notion of a context.

A memory cell $i$ has an associated state at time $t$.

\ac{LSTM} has been successfully applied to guitar amplifier \cite{Wright2019,Wrightetal2020} and time-varying effects modeling \cite{Wright2020}.

%%%%%%%%%%%%%%%%%%%%%%%%%%%%%%%%%%%%%%%%%%%%%%%%%%%%%%%%%%%%%%%%%%%%%%%%%%%%%%%%%%%%%%%
\subsection*{Residual Networks}
%%%%%%%%%%%%%%%%%%%%%%%%%%%%%%%%%%%%%%%%%%%%%%%%%%%%%%%%%%%%%%%%%%%%%%%%%%%%%%%%%%%%%%%

% TODO: Add figure with a ResNet
Another architecture significant to this work is the \ac{ResNet} \cite{He2015}. Its characteristic feature is the so-called \emph{\ac{ResBlock}} containing a \emph{skip connection}. The skip connection provides a direct path from the input to the output which allows for summing the input with the network's output. It effectively makes the network learn not the mapping $f^*(\pmb{x})$ but the residual $h^*(\pmb{x}) = f^*(\pmb{x}) - \pmb{x}$. On the one hand, it alleviates the problem of the \emph{vanishing gradient} \cite{Goodfellow-et-al-2016} and eventually allows for training very deep architectures with hundreds of layers \cite{He2015}. Depth is one of the most significant factors in neural network design, thus, being able to train deeper and deeper architectures allows for improving performance on the desired tasks. On the other hand, the skip connection can simplify the training process; instead of learning a complete function, the network must only learn its residual. For example, if the true mapping is the identity function, it is easier to learn the residual (which is $0$ everywhere) rather than the identity itself, which would involve approximating a hyperplane with nonlinear units \cite{He2015}.


%%%%%%%%%%%%%%%%%%%%%%%%%%%%%%%%%%%%%%%%%%%%%%%%%%%%%%%%%%%%%%%%%%%%%%%%%%%%%%%%%%%%%%%
\section*{Training of Neural Networks}
%%%%%%%%%%%%%%%%%%%%%%%%%%%%%%%%%%%%%%%%%%%%%%%%%%%%%%%%%%%%%%%%%%%%%%%%%%%%%%%%%%%%%%%

Training of \acp{ANN} is done using an optimization algorithm from the \emph{gradient descent} family of algorithms \cite{Goodfellow-et-al-2016}. In general, gradient descent methodology requires specifying a \emph{loss} or a \emph{target function} between the network's output and the desired output. The optimization then proceeds by sampling a \emph{minibatch} of examples from the \emph{training set} in each iteration, processing them with the network, and calculating the loss. The loss function is a measure of dissimilarity (or distance) between the network's output and the \emph{target}. For example, the training set may consist of images (inputs) with associated labels (targets). The network should learn the mapping from an image to its label. After calculating the loss, the gradient of the loss with respect to the network's parameters (mainly weights and biases) is computed via \emph{backpropagation} through the network using the chain rule of derivatives. After calculating the gradient of the loss with respect to each individual parameter, a step in the direction of the negated gradient is performed (steepest descent). The size of the step is determined by the \emph{learning rate}. Some algorithms use a different step size for each parameter, for example, Adam \cite{Kingma2017}.

%%%%%%%%%%%%%%%%%%%%%%%%%%%%%%%%%%%%%%%%%%%%%%%%%%%%%%%%%%%%%%%%%%%%%%%%%%%%%%%%%%%%%%%
\subsection*{Loss Functions}
%%%%%%%%%%%%%%%%%%%%%%%%%%%%%%%%%%%%%%%%%%%%%%%%%%%%%%%%%%%%%%%%%%%%%%%%%%%%%%%%%%%%%%%

In order to train, test, and compare machine learning models, we need a performance measure \cite{Goodfellow-et-al-2016}. For supervised learning problems, i.e., problems where the targets of the training set are known, a natural approach is to measure the error between the network's output and the target values. One of the most popular metrics is the \ac{MSE}

\begin{equation}
  \mathcal{E}_\text{MSE}(y, \hat{y}) = \frac{1}{N} \sum \limits_{n=0}^{N-1} (y[n] - \hat{y}[n])^2,
\end{equation}
where $y$ is the target signal, $\hat{y}$ is the network's output and $N$ is the length of these two signals.

In the context of audio processing, the \ac{MSE} loss has the drawback that it prioritizes errors of signals with large dynamics and neglects the inaccuracies of quiet parts \cite{Parker2019}. These quiet parts are often perceptually relevant. Therefore, a relative error measure is thought to be more suitable in audio applications.

A commonly used relative error measure in \ac{VA} modeling is the \ac{ESR} \cite{Wright2019,Wright2019a, Wrightetal2020,Wright2020}

\begin{equation}
  \mathcal{E}_\text{ESR}(y, \hat{y}) = \frac{\sum_{n=0}^{N-1} (y[n] - \hat{y}[n])^2}{\sum_{n=0}^{N-1} (y[n])^2},
\end{equation}
i.e., the energy of the error signal divided by the energy of the target signal.

In order to suppress the DC offset in the network's output a DC loss term may be added to the loss function \cite{Wright2019a,Wright2020}
\begin{equation}
  \mathcal{E}_\text{DC}(y, \hat{y}) = \frac{\left(\frac{1}{N} \sum_{n=0}^{N-1} (y[n] - \hat{y}[n])\right)^2}{\frac{1}{N} \sum_{n=0}^{N-1} (y[n])^2}.
\end{equation}

Additionally, it is important to take the perceptual aspect into account by penalizing errors that result in less perceptually convincing results. This can be achieved using a \emph{pre-emphasis filter}, i.e., a filter that boosts the perceptually relevant frequencies and suppresses the irrelevant ones.

A first-order highpass filter can be used as a pre-emphasis filter as shown in \cite{Wright2019,Wrightetal2020,Wright2020}. Such a filter has a transfer function of the form

\begin{equation}
  H(z) = 1 - 0.85 z^{-1}.
\end{equation}

This form of pre-emphasis is straightforward to implement, achieves perceptually superior performance to non-preemphasized loss functions, and is just slightly inferior to much more complicated pre-emphasis filters \cite{Wright2019a}.

After \cite{Wright2019a,Wright2020,Wright2019}, this works uses the following combined loss function for neural network training

\begin{equation}
  \mathcal{E}(y, \hat{y}) = \mathcal{E}_\text{ESR}(y_\text{p}, \hat{y}_\text{p}) + \mathcal{E}_\text{DC}(y, \hat{y}),
  \label{eq:final_loss_function}
\end{equation}
where "p" in the subscript marks signals that were pre-emphasized.

%%%%%%%%%%%%%%%%%%%%%%%%%%%%%%%%%%%%%%%%%%%%%%%%%%%%%%%%%%%%%%%%%%%%%%%%%%%%%%%%%%%%%%%
\subsection*{Teacher Forcing, Curriculum Learning, and Scheduled Sampling}
%%%%%%%%%%%%%%%%%%%%%%%%%%%%%%%%%%%%%%%%%%%%%%%%%%%%%%%%%%%%%%%%%%%%%%%%%%%%%%%%%%%%%%%

% TODO: Add a figure on teacher forcing
% TODO: Cite Goodfellow and "Scheduled Sampling(...)"

A sequence-to-sequence \ac{RNN} during processing is using its previous output, previous hidden state, and current input to produce an output at a single time step. If the network's outputs diverge from the target outputs introducing an error, the error will accumulate and eventually lead the network to a completely wrong part of the solution space \cite{Goodfellow-et-al-2016}. To prevent this, one may supply the true target of the previous time step as the input at the current time step. Since the previous output is still accounted for in the loss function, the gradient will continue account for it, however, the network will be more "guided" during training. The act of supplying the ground truth output at time step $t$ as an input at time step $t+1$ of an \ac{RNN} is called \emph{teacher forcing} \cite{Goodfellow-et-al-2016}.

The guidance provided by teacher forcing is especially important in the initial phase of the training, where the network's output has still relatively high validation error. Ultimately, during test, the network should operate using only self-produced outputs (without ground truth knowledge). Therefore, the better the performance of the network during training the less guidance should be provided resulting in no guidance at test time. Decreasing the amount of ground truth output supply throughout training is called \emph{curriculum learning}. The decrease can be fixed or random. In the latter case, for each minibatch, we "flip the coin" to determine whether teacher forcing should be used. "Flipping the coin" may be implemented as sampling from a Bernoulli distribution with parameter $\epsilon$ determining the probability of using teacher forcing. For example, at test time, we have $\epsilon=0$. This approach bears the name of \emph{scheduled sampling} \cite{Bengio2015}. Again, the $\epsilon$ parameter may be decreased with each minibatch. In this work, we only consider a linear decay of $\epsilon$.

%%%%%%%%%%%%%%%%%%%%%%%%%%%%%%%%%%%%%%%%%%%%%%%%%%%%%%%%%%%%%%%%%%%%%%%%%%%%%%%%%%%%%%%
\section*{Software for Deep Learning}
%%%%%%%%%%%%%%%%%%%%%%%%%%%%%%%%%%%%%%%%%%%%%%%%%%%%%%%%%%%%%%%%%%%%%%%%%%%%%%%%%%%%%%%

A deep learning practitioner needs to implement the designed architectures, loss functions, dataset handling, training and test procedures, etc. For this purpose, we used the PyTorch library \cite{Pytorch}. Additional functionality was provided using \cite{CoreAudioML}.

%%%%%%%%%%%%%%%%%%%%%%%%%%%%%%%%%%%%%%%%%%%%%%%%%%%%%%%%%%%%%%%%%%%%%%%%%%%%%%%%%%%%%%%
\subsection{Neural Ordinary Differential Equations}
\label{sec:neural_odes}
%%%%%%%%%%%%%%%%%%%%%%%%%%%%%%%%%%%%%%%%%%%%%%%%%%%%%%%%%%%%%%%%%%%%%%%%%%%%%%%%%%%%%%%

%TODO: Add citations
%TODO: Add ODENet figure

It has been observed that learning the residual instead of the function can be viewed as an Euler step of solving a differential equation. In the Euler method of solving \acp{ODE}, function's value at time step $t_i$ is approximated by adding the value of the derivative at time step $t_{i-1}$ multiplied by the time step size $t_i - t_{i-1}$ to the function's value at $t_i$. \cite{Chen2018} stated this observation explicitly and proposed to change the discrete scheme to a continuous one. In such a scheme, the network learns not the residual but the derivative of the desired function. This "derivative network" is supplied to a numerical solver of choice. The loss is calculated using the solver's (not the network's) output. Such generically defined architecture was named ODENet. The backpropagation through the solver's operations can be done in constant time using the adjoint sensitivity method \cite{Chen2018}.

\cite{Karlsson2019} successfully applied ODENet to model dynamical systems, i.\ e., systems governed by \acp{ODE}. 

%%%%%%%%%%%%%%%%%%%%%%%%%%%%%%%%%%%%%%%%%%%%%%%%%%%%%%%%%%%%%%%%%%%%%%%%%%%%%%%%%%%%%%%
\section{Virtual Analog Modeling}
\label{sec:virtual_analog_modeling}
%%%%%%%%%%%%%%%%%%%%%%%%%%%%%%%%%%%%%%%%%%%%%%%%%%%%%%%%%%%%%%%%%%%%%%%%%%%%%%%%%%%%%%%
\Ac{VA} effects are digital emulations of audio systems that were originally built using analog electronic or electromechanical components \cite{Zoelzer2011}. They arose out of the demand for the reproduction of characteristic tonal distortions of these devices but with the digital stability and the ease of integration with existing software. Devices modelled range from filters, to time-varying effects, amplifiers, mechanical reverb units, and tape or vinyl distortions.

One can distinguish three general approaches to \ac{VA} modeling \cite{Kiiski2016,Wright2020}. In \emph{black-box} modelling, only the input-output relation of a system is examined and a signal model is constructed to mimic that behavior. Neural networks have successfully been applied to this kind of modeling for guitar amplifiers \cite{Wright2019,Wrightetal2020}. In \emph{white-box} modelling an internal structure of the system under study is examined and used to construct an algorithm aiming at reproducing its behavior. Sometimes it is referred to as \emph{physical modeling}. Typical approaches in this category are a numerical solution of \acp{ODE} derived from electronic circuit analysis \cite{Yeh2007,Eichas2014} or wave-digital filters \cite{PASPWEB2010}. \emph{Grey-box} modeling falls somewhere in between the two already mentioned approaches, where we use some knowledge about the inner workings of the device under study to design a model and, subsequently, take advantage of the input and output data to adjust the model's parameters. This approach has been successfully applied to time-varying effects modeling, like a phaser or a flanger \cite{Kiiski2016,Wright2020}, where a \ac{LFO} signal is estimated and used for conditioning the model.
