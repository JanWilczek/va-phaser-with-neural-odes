
%%%%%%%%%%%%%%%%%%%%%%%%%%%%%%%%%%%%%%%%%%%%%%%%%%%%%%%%%%%%%%%%%%%%%%%%%%%%%%%%%%%%%%%
\chapter{Phaser}
\label{chap:phaser}
%%%%%%%%%%%%%%%%%%%%%%%%%%%%%%%%%%%%%%%%%%%%%%%%%%%%%%%%%%%%%%%%%%%%%%%%%%%%%%%%%%%%%%%
%TODO add a figure of the phaser
Phaser is a filter-based, time-varying effect \cite{Zoelzer2011}. It applies a series of notches to the spectrum of the input signal. The center frequencies of these notches are controlled by a \ac{LFO} and constantly vary with respect to one another, each moving up and down the frequency range. Phaser is typically implemented by summing the input signal processed by a series of notch or allpass filters with a direct path \cite{Zoelzer2011}. Additionally, a feedback connection can be added around the filter chain \cite{Kiiski2016}. The feedback connection makes the phaser behavior more complex by introducing resonances and non-notch frequency boosting.
