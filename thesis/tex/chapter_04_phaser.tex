
%%%%%%%%%%%%%%%%%%%%%%%%%%%%%%%%%%%%%%%%%%%%%%%%%%%%%%%%%%%%%%%%%%%%%%%%%%%%%%%%%%%%%%%
\chapter{Phaser}
\label{chap:phaser}
%%%%%%%%%%%%%%%%%%%%%%%%%%%%%%%%%%%%%%%%%%%%%%%%%%%%%%%%%%%%%%%%%%%%%%%%%%%%%%%%%%%%%%%
Phaser is a filter-based, time-varying effect \cite{Zoelzer2011}. It applies a series of notches to the spectrum of the input signal. The center frequencies of these notches are controlled by a \ac{LFO} and vary periodically over time, moving up and down the frequency range. The time-varying notches create a "swooshing" sound that is often identified as "robotic".

Phaser should not be confused with another notch-sweeping effect, namely, flanger. Flanging differs from phasing in that the time-varying notches are equally-spaced and seemingly infinite in number (limited only by the sampling rate), whereas in phasing the notches are nonuniformly spaced and their number is fixed. In flanging, the notches are created by summing the direct path with an output of a variable-length delay line, whose length is controlled by an \ac{LFO}. The result is a time-varying comb filter. Its sound is often compared to a jet taking off.


%%%%%%%%%%%%%%%%%%%%%%%%%%%%%%%%%%%%%%%%%%%%%%%%%%%%%%%%%%%%%%%%%%%%%%%%%%%%%%%%%%%%%%%
\section{Phasing Effect}
\label{sec:phasing_effect}
%%%%%%%%%%%%%%%%%%%%%%%%%%%%%%%%%%%%%%%%%%%%%%%%%%%%%%%%%%%%%%%%%%%%%%%%%%%%%%%%%%%%%%%

This section explains the basics of phasing and a typical implementation of a phaser using allpass filters. Firstly, a high-level view is presented followed by an explanation of an allpass filter and its impact on the phaser's behavior. Finally, known digital phaser models are presented.

%%%%%%%%%%%%%%%%%%%%%%%%%%%%%%%%%%%%%%%%%%%%%%%%%%%%%%%%%%%%%%%%%%%%%%%%%%%%%%%%%%%%%%%
\subsection{Overview}
\label{sec:phaser_overview}
%%%%%%%%%%%%%%%%%%%%%%%%%%%%%%%%%%%%%%%%%%%%%%%%%%%%%%%%%%%%%%%%%%%%%%%%%%%%%%%%%%%%%%%

\begin{figure}
    \centering
    \begin{tikzpicture}
    \node[dspnodeopen,dsp/label=above]              (x) {$x[n]$};
    \node[dspnodefull, right=of x]              (d0) {};
    \node[coordinate,below=of d0,yshift=-0.5cm] (ap0) {};
    
    \foreach \i in {1,...,4}
    {
        \pgfmathtruncatemacro{\im}{\i - 1};
        \node[dspsquare,minimum size=1cm,right=of ap\im] (ap\i) {AP\textsubscript{\i}};
        \draw[dspconn] (ap\im) -- (ap\i);
    }
    
    \node[dspmultiplier,right=of ap4] (wet) {$W$};
    \node[dspmultiplier,above=of wet,yshift=1mm] (dry) {$D$};
    \node[coordinate,right=of wet] (c3) {};
    \node[dspadder,above=of c3,yshift=3mm] (adder) {};
    \node[dspnodeopen,right=of adder,dsp/label=above] (fx) {$y[n]$};

    \draw[dspline] (x) -- (d0);
    \draw[dspline] (d0) -- (ap0);
    \draw[dspconn] (ap4) -- (wet);
    \draw[dspline] (wet) -- (c3);
    \draw[dspconn] (c3) -- (adder);
    \draw[dspconn] (d0) -- (dry);
    \draw[dspconn] (dry) -- (adder);
    \draw[dspconn] (adder) -- (fx);
\end{tikzpicture}

    \caption{A phaser implementation.}
    \label{fig:phaser}
\end{figure}

Phaser is typically implemented by summing the input signal processed by a series of notch or allpass filters with a direct path \cite{Zoelzer2011}. Additionally, a feedback connection can be added around the filter chain \cite{Kiiski2016}. The feedback connection makes the phaser behavior more complex by introducing resonances and non-notch frequency boosting.

In \Figure{fig:phaser}, a basic diagram of a phaser is depicted. This schematic holds for analog as well as digital phasers \cite{PASPWEB2010}. The direct path consists of a simple scaling by $D$ ("dry" signal). The indirect path is a series of allpass filters followed by a scaling by $W$ ("wet" signal). The allpass filters introduce frequency-dependent delay. Due to these shifts, summation of the two paths causes destructive and constructive interference, resulting in notches and peaks in the magnitude spectrum of the phaser. Scaling factors $D$ and $W$ control the amount of interference. The deepest possible notches appear when $D = W$. To keep the amplitude constant, one typically couples $D$ and $W$ so that $D + W = 1$.

To understand phaser's effect on the signal, one should examine the allpass filters first. That is the topic of the next section.

% TODO: Insert a spectrogram of phaser's effect on white noise

%%%%%%%%%%%%%%%%%%%%%%%%%%%%%%%%%%%%%%%%%%%%%%%%%%%%%%%%%%%%%%%%%%%%%%%%%%%%%%%%%%%%%%%
\subsection{First-Order Allpass Filter}
\label{sec:first_order_allpass_filter}
%%%%%%%%%%%%%%%%%%%%%%%%%%%%%%%%%%%%%%%%%%%%%%%%%%%%%%%%%%%%%%%%%%%%%%%%%%%%%%%%%%%%%%%
\begin{figure}
    \centering
    \input{figures/tikz/first_order_allpass_filter.tex}
    \caption{First-order digital allpass filter.}
    \label{fig:first_order_allpass_filter}
\end{figure}

An allpass filter is a filter with unit magnitude gain at all frequencies. The general digital allpass filter has the following transfer function \cite{Zoelzer2011,PASPWEB2010,Kiiski2016}

\begin{equation}
    H_\text{allpass}(z) = \frac{a_1 + z^{-1}}{1 + a_1 z^{-1}},
\end{equation}
where $a_1$ coefficient determines the break frequency of the allpass, i.e., the frequency at which the phase shift is exactly $\frac{\pi}{2}$ $\frac{\text{rad}}{\text{s}}$. Given the break frequency $f_c$ in Hz and the sampling rate $f_s$ in Hz, the $a_1$ coefficient can be calculated as

\begin{equation}
    a_1 = \frac{\tan (\pi f_c / f_s) - 1}{\tan(\pi f_c / f_s) + 1}.
    \label{eq:allpass_coefficient}
\end{equation}

\Equation{eq:allpass_coefficient} is the result of applying the bilinear transform to the transfer function of an analog allpass

\begin{equation}
    H_{\text{allpass}}^{\text{a}} (s) = \frac{s - 2\pi f_c}{s + 2\pi f_c},
\end{equation}
where 'a' superscript marks the analog (continuous) domain of the transfer function.

% TODO: Insert phase response formula

In \Figure{fig:phase_response_allpass_filter} the phase responses of a digital allpass filters with different break frequencies are shown. An allpass does not change the magnitude of any frequency and it introduces a frequency-dependent delay that varies from 0 at DC to $\pi$ at $\frac{f_c}{2}$. 

\begin{figure}
    \centering
    % \documentclass[preview,border=4mm,convert={density=600,outext=.png}]{standalone}
 
% \usepackage{url}
% \usepackage{tikz}
% \usepackage{pgfplots}

% \begin{document}
\begin{tikzpicture}
    \pgfmathsetmacro{\wb}{pi/8} % in radians, 2 * pi * fb / fs
    \pgfmathsetmacro{\a}{-(1 - tan(deg(\wb/2)))/(1+tan(deg(\wb/2)))};
    \pgfmathsetmacro{\ad}{-(1 - tan(deg(2*\wb/2)))/(1+tan(deg(2*\wb/2)))};
    \pgfmathsetmacro{\at}{-(1 - tan(deg(3*\wb/2)))/(1+tan(deg(3*\wb/2)))};
    \begin{axis}[
        domain=0.00001:pi,
        axis lines = middle,
        xmode=log,
    ]
        % \addplot[smooth,mark=none,color=red] {- 2 * rad(atan(x/\wb))}; % Smith, analog
        % \addplot[smooth,mark=none,color=blue] {rad(atan(((\a*\a - 1) * sin(deg(x))) / (2*\a + (1 + \a*\a) * cos(deg(x)))))}; % mine
        
        \addplot[smooth,mark=none,color=black] {-x + 2 * rad(atan((\a * sin(deg(x)) / (1 + \a * cos(deg(x))))))}; % Kiiski et al.
        \addplot[smooth,mark=none,color=black,dashed] {-x + 2 * rad(atan((\ad * sin(deg(x)) / (1 + \ad * cos(deg(x))))))}; % Kiiski et al.
        \addplot[smooth,mark=none,color=black,dotted] {-x + 2 * rad(atan((\at * sin(deg(x)) / (1 + \at * cos(deg(x))))))}; % Kiiski et al.

        
        % \addplot[smooth,mark=none,color=black] {- 2 * rad(atan(x/0.1))};
        % \addplot[smooth,mark=none,color=black,dashed] {- 2 * rad(atan(x/0.2))};
        % \addplot[smooth,mark=none,color=black,dotted] {- 2 * rad(atan(x/0.3))};
    \end{axis}
\end{tikzpicture}
% \end{document}

    \caption{Phase response of a first-order allpass filter.}
    \label{fig:phase_response_allpass_filter}
\end{figure}

%%%%%%%%%%%%%%%%%%%%%%%%%%%%%%%%%%%%%%%%%%%%%%%%%%%%%%%%%%%%%%%%%%%%%%%%%%%%%%%%%%%%%%%
\subsection{Analog and Digital Phasers}
%%%%%%%%%%%%%%%%%%%%%%%%%%%%%%%%%%%%%%%%%%%%%%%%%%%%%%%%%%%%%%%%%%%%%%%%%%%%%%%%%%%%%%%

Cascading \ac{LTI} systems results in the summation of their phase responses \cite{Oppenheim1997}. Thus, a cascade of $N$ allpass filters introduces a frequency dependent delay that varies from 0 at DC to $N\pi$ at $\frac{f_c}{2}$ \cite{PASPWEB2010}. Summing the cascaded filters with the direct path (as in \Figure{fig:phaser}) leads to the appearance of notches in the output spectrum of a phaser at frequencies for which the phase shift is an odd multiplicity of $\pi$, i.e., $\pi, 3\pi, 5\pi$, etc. Since each first-order allpass filter has a maximum phase shift of $\pi$, $N$ allpass filters result in $\frac{N}{2}$ notches (assuming that $N$ is even).

The core idea of a phaser is to modulate the break frequencies of the allpass filters so that the notches sweep up and down the frequency range creating an audibly interesting effect. In analog phasers, that are implemented using analog electronic circuitry, the values of physical components may be modulated. In digital phasers, the implementers can decide whether to modulate the break frequencies directly or to modulate just the allpass coefficients \cite{Kiiski2016}. The difference between these two design decisions are shown in the .%TODO: Add pink spectrum figure.

\begin{figure}
    \begin{subfigure}{0.33\textwidth}
        \centering
        \input{figures/tikz/pink_noise_10s_stft.tex}
        % \includegraphics[width=\textwidth]{figures/png/pink_noise_10s}
    \end{subfigure}
    \begin{subfigure}{0.33\textwidth}
        \centering
        % \includegraphics[width=\textwidth]{figures/png/pink_noise_10s}
        \input{figures/tikz/pink_noise_10s_stft.tex}
    \end{subfigure}
    \begin{subfigure}{0.33\textwidth}
        \centering
        % \includegraphics[width=\textwidth]{figures/png/pink_noise_10s}
        \input{figures/tikz/pink_noise_10s_stft.tex}
    \end{subfigure}
\end{figure}

%%%%%%%%%%%%%%%%%%%%%%%%%%%%%%%%%%%%%%%%%%%%%%%%%%%%%%%%%%%%%%%%%%%%%%%%%%%%%%%%%%%%%%%
\subsection{Phaser Modeling}
%%%%%%%%%%%%%%%%%%%%%%%%%%%%%%%%%%%%%%%%%%%%%%%%%%%%%%%%%%%%%%%%%%%%%%%%%%%%%%%%%%%%%%%

%%%%%%%%%%%%%%%%%%%%%%%%%%%%%%%%%%%%%%%%%%%%%%%%%%%%%%%%%%%%%%%%%%%%%%%%%%%%%%%%%%%%%%%
\section{Training Data}
\label{sec:phaser_training_data}
%%%%%%%%%%%%%%%%%%%%%%%%%%%%%%%%%%%%%%%%%%%%%%%%%%%%%%%%%%%%%%%%%%%%%%%%%%%%%%%%%%%%%%%

%%%%%%%%%%%%%%%%%%%%%%%%%%%%%%%%%%%%%%%%%%%%%%%%%%%%%%%%%%%%%%%%%%%%%%%%%%%%%%%%%%%%%%%
\section{Training}
\label{sec:phaser_training}
%%%%%%%%%%%%%%%%%%%%%%%%%%%%%%%%%%%%%%%%%%%%%%%%%%%%%%%%%%%%%%%%%%%%%%%%%%%%%%%%%%%%%%%

%%%%%%%%%%%%%%%%%%%%%%%%%%%%%%%%%%%%%%%%%%%%%%%%%%%%%%%%%%%%%%%%%%%%%%%%%%%%%%%%%%%%%%%
\section{Compared Models}
\label{sec:phaser_models}
%%%%%%%%%%%%%%%%%%%%%%%%%%%%%%%%%%%%%%%%%%%%%%%%%%%%%%%%%%%%%%%%%%%%%%%%%%%%%%%%%%%%%%%

%%%%%%%%%%%%%%%%%%%%%%%%%%%%%%%%%%%%%%%%%%%%%%%%%%%%%%%%%%%%%%%%%%%%%%%%%%%%%%%%%%%%%%%
\section{Results and Discussion}
\label{sec:phaser_results}
%%%%%%%%%%%%%%%%%%%%%%%%%%%%%%%%%%%%%%%%%%%%%%%%%%%%%%%%%%%%%%%%%%%%%%%%%%%%%%%%%%%%%%%

