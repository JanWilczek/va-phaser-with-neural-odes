
%%%%%%%%%%%%%%%%%%%%%%%%%%%%%%%%%%%%%%%%%%%%%%%%%%%%%%%%%%%%%%%%%%%%%%%%%%%%%%%%%%%%%%%
\section{Phaser}
\label{chap:phaser}
%%%%%%%%%%%%%%%%%%%%%%%%%%%%%%%%%%%%%%%%%%%%%%%%%%%%%%%%%%%%%%%%%%%%%%%%%%%%%%%%%%%%%%%
Phaser is a filter-based, time-varying effect \cite{Zoelzer2011}. It applies a series of notches to the spectrum of the input signal. The center frequencies of these notches are controlled by an \ac{LFO} and vary periodically over time, moving up and down the frequency range. The time-varying notches create a "swooshing" sound that is often identified as "robotic".

Phaser should not be confused with another notch-sweeping effect, namely, flanger. Flanging differs from phasing in that the time-varying notches are equally-spaced and seemingly infinite in number (limited only by the sampling rate), whereas in phasing the notches are nonuniformly spaced and their number is fixed. In flanging, the notches are created by summing the direct path with an output of a variable-length delay line, whose length is controlled by an \ac{LFO}. The result is a time-varying comb filter. Its sound is often compared to a jet taking off.

This section explains the basics of phasing and presents an implementation of the phaser using allpass filters. Firstly, a high-level view is presented followed by an explanation of an allpass filter and its impact on the phaser's behavior. Finally, known digital phaser models are presented.

%%%%%%%%%%%%%%%%%%%%%%%%%%%%%%%%%%%%%%%%%%%%%%%%%%%%%%%%%%%%%%%%%%%%%%%%%%%%%%%%%%%%%%%
\subsection{Overview}
\label{sec:phaser_overview}
%%%%%%%%%%%%%%%%%%%%%%%%%%%%%%%%%%%%%%%%%%%%%%%%%%%%%%%%%%%%%%%%%%%%%%%%%%%%%%%%%%%%%%%

\begin{figure}
    \centering
    \input{figures/svg/allpass_phaser.pdf_tex}
    \caption{A phaser implementation using allpass filters.}
    \label{fig:phaser}
\end{figure}

Phaser is typically implemented by summing the input signal processed by a series of notch or allpass filters with a direct path \cite{Zoelzer2011}. The summation introduces notches in the magnitude frequency spectrum of the output signal. Additionally, a feedback connection can be added around the filter chain \cite{Kiiski2016}. The feedback connection makes the phaser behavior more complex by introducing resonances and non-notch frequency boosting.

In \Figure{fig:phaser}, a basic diagram of an allpass-based phaser is depicted. This schematic holds for analog as well as digital phasers \cite{PASPWEB2010}. The direct path consists of a simple scaling of the input signal $x[n]$ by $d$ ("dry" signal). The indirect path is a series of allpass filters AP\textsubscript{1}, \dots, AP\textsubscript{N} modulated by an \ac{LFO} signal $s_\text{LFO}[n]$ and followed by a scaling by $w$ ("wet" signal). Here, the \ac{LFO} generates a rectified sine signal. The allpass filters introduce a frequency-dependent delay. Due to these shifts, the summation of the two paths causes destructive and constructive interference, resulting in notches and peaks in the magnitude frequency spectrum of the phaser. Modulation of allpass filters causes the modulation of the notches in the spectrum. Scaling factors $d$ and $w$ control the amount of interference. The deepest possible notches appear when $d = w$. To keep the amplitude constant, one typically couples $d$ and $w$ so that $d + w = 1$.

To understand phaser's effect on the signal, one should examine the allpass filters first. That is the topic of the next section.

%%%%%%%%%%%%%%%%%%%%%%%%%%%%%%%%%%%%%%%%%%%%%%%%%%%%%%%%%%%%%%%%%%%%%%%%%%%%%%%%%%%%%%%
\subsection{First-Order Allpass Filter}
\label{sec:first_order_allpass_filter}
%%%%%%%%%%%%%%%%%%%%%%%%%%%%%%%%%%%%%%%%%%%%%%%%%%%%%%%%%%%%%%%%%%%%%%%%%%%%%%%%%%%%%%%
\begin{figure}
    \centering
    \vspace{2mm}
    \begin{tikzpicture}
    \node[dspnodeopen,dsp/label=above]              (x) {$x[n]$};
    \node[dspadder, right=of x]              (adder1) {};
    \node[dspnodefull, right=of adder1]              (c0) {};
    \node[coordinate,above=of c0] (c1) {};
    \node[dspmixer,right=of c1] (a1_forward) {$a_1$};
    \node[coordinate,right=of a1_forward] (c3) {};
    
    % Unit delay
    \node[dspsquare,right=of c0] (delay) {$z^{-1}$};
    \node[dspnodefull,right=of delay] (c4) {};
    \node[dspadder,right=of c4] (adder) {};
    \node[dspnodeopen,right=of adder,dsp/label=above] (y) {$y[n]$};
    \node[coordinate, below=of c4] (c5) {};
    \node[dspmixer,left= of c5,dsp/label=below] (a1_feedback) {$a_1$};
    \node[coordinate,left=of a1_feedback] (c6) {};

    % Feedforward connections
    \draw[dspconn] (x) -- (adder1);
    \draw[dspline] (adder1) -- (c0);
    \draw[dspline] (c0) -- (c1);
    \draw[dspconn] (c1) -- (a1_forward);
    \draw[dspline] (a1_forward) -- (c3);
    \draw[dspconn] (c3) -- (adder);


    \draw[dspconn] (c0) -- (delay);
    \draw[dspline] (delay) -- (c4);
    \draw[dspconn] (c4) -- (adder);
    \draw[dspconn] (adder) -- (y);

    % Feedback connections
    \draw[dspline] (c4) -- (c5);
    \draw[dspconn] (c5) -- (a1_feedback);
    \draw[dspline] (a1_feedback) -- (c6);
    \draw[dspconn] (c6) -- (adder1);
\end{tikzpicture}

    \caption{First-order digital allpass filter.}
    \label{fig:first_order_allpass_filter}
\end{figure}

An allpass filter is a filter with unit magnitude gain at all frequencies. The general digital allpass filter has the following transfer function \cite{Zoelzer2011,PASPWEB2010,Kiiski2016}

\begin{equation}
    H_\text{allpass}(z) = \frac{a_1 + z^{-1}}{1 + a_1 z^{-1}},
    \label{eq:allpass_transfer_function}
\end{equation}
where $a_1$ coefficient determines the break frequency of the allpass, i.e., the frequency at which the phase shift is exactly $-\frac{\pi}{2}$ $\frac{\text{rad}}{\text{s}}$. Given the break frequency $f_c$ in Hz and the sampling rate $f_s$ in Hz, the $a_1$ coefficient can be calculated as

\begin{equation}
    a_1 = \frac{\tan (\pi f_c / f_s) - 1}{\tan(\pi f_c / f_s) + 1}.
    \label{eq:allpass_coefficient}
\end{equation}

\Equation{eq:allpass_coefficient} is the result of applying the bilinear transform to the transfer function of an analog allpass

\begin{equation}
    H_{\text{allpass}}^{\text{a}} (s) = \frac{s - 2\pi f_c}{s + 2\pi f_c},
\end{equation}
where 'a' superscript marks the analog (continuous) domain of the transfer function.

The difference equation resulting from \Equation{eq:allpass_transfer_function} is
\begin{equation}
    y[n] = a_1 x[n] + \underbrace{x[n-1] - a_1 y[n-1]}_{d[n-1]},
    \label{eq:allpass_filter_difference_equation}
\end{equation}
where $x[n]$ denotes the input signal to the phaser and $y[n]$ denotes the output signal of the phaser. A diagram depicting this equation is shown in \Figure{fig:first_order_allpass_filter}. Two rightmost addends on the right-hand side of \Equation{eq:allpass_filter_difference_equation} (denoted by $d[n-1]$) can be implemented using a one-element delay line.

% TODO Explain shortly what you mean by the phase response (argument of the complex transfer function)

The phase response of an allpass filter (specified in the \SIrange{-\pi}{0}{rad} range) is given by the formula \cite{Kiiski2016}

\begin{equation}
    \theta (\omega) = - \omega + 2 \arctan \left( \frac{a_1 \sin \omega}{1 + a_1 \cos \omega} \right),
\end{equation}
where $\omega = 2 \pi f / f_s$ is the angular frequency given in radians.

In \Figure{fig:phase_response_allpass_filter}, the phase responses of digital allpass filters with different break frequencies $\omega_\text{b}$ are shown. The break frequency is a frequency at which the phase shift is equal to \SI{-\pi/2}{rad}. An allpass does not change the magnitude of any frequency and it introduces a frequency-dependent delay that varies from 0 at DC to $-\pi$ at $f = \frac{f_s}{2}$ (or, equivalently, $\omega = \pi$). 

\begin{figure}
    \centering
    % \documentclass[preview,border=4mm,convert={density=600,outext=.png}]{standalone}
 
% \usepackage{url}
% \usepackage{tikz}
% \usepackage{pgfplots}

% \begin{document}
\begin{tikzpicture}
    \pgfmathsetmacro{\wb}{pi/8} % in radians, 2 * pi * fb / fs
    \pgfmathsetmacro{\a}{-(1 - tan(deg(\wb/2)))/(1+tan(deg(\wb/2)))};
    \pgfmathsetmacro{\ad}{-(1 - tan(deg(2*\wb/2)))/(1+tan(deg(2*\wb/2)))};
    \pgfmathsetmacro{\at}{-(1 - tan(deg(3*\wb/2)))/(1+tan(deg(3*\wb/2)))};
    \begin{axis}[
        domain=0.00001:pi,
        axis lines = middle,
        xmode=log,
    ]
        % \addplot[smooth,mark=none,color=red] {- 2 * rad(atan(x/\wb))}; % Smith, analog
        % \addplot[smooth,mark=none,color=blue] {rad(atan(((\a*\a - 1) * sin(deg(x))) / (2*\a + (1 + \a*\a) * cos(deg(x)))))}; % mine
        
        \addplot[smooth,mark=none,color=black] {-x + 2 * rad(atan((\a * sin(deg(x)) / (1 + \a * cos(deg(x))))))}; % Kiiski et al.
        \addplot[smooth,mark=none,color=black,dashed] {-x + 2 * rad(atan((\ad * sin(deg(x)) / (1 + \ad * cos(deg(x))))))}; % Kiiski et al.
        \addplot[smooth,mark=none,color=black,dotted] {-x + 2 * rad(atan((\at * sin(deg(x)) / (1 + \at * cos(deg(x))))))}; % Kiiski et al.

        
        % \addplot[smooth,mark=none,color=black] {- 2 * rad(atan(x/0.1))};
        % \addplot[smooth,mark=none,color=black,dashed] {- 2 * rad(atan(x/0.2))};
        % \addplot[smooth,mark=none,color=black,dotted] {- 2 * rad(atan(x/0.3))};
    \end{axis}
\end{tikzpicture}
% \end{document}

    \caption{Phase response of a first-order allpass filter.}
    \label{fig:phase_response_allpass_filter}
\end{figure}

%%%%%%%%%%%%%%%%%%%%%%%%%%%%%%%%%%%%%%%%%%%%%%%%%%%%%%%%%%%%%%%%%%%%%%%%%%%%%%%%%%%%%%%
\subsection{Analog and Digital Phasers}
%%%%%%%%%%%%%%%%%%%%%%%%%%%%%%%%%%%%%%%%%%%%%%%%%%%%%%%%%%%%%%%%%%%%%%%%%%%%%%%%%%%%%%%

Cascading \ac{LTI} systems results in a summation of their phase responses \cite{Oppenheim1997}. Thus, a cascade of $N$ allpass filters introduces a frequency dependent delay that varies from 0 at DC to $-N\pi$ at $\frac{f_c}{2}$ \cite{PASPWEB2010}. Summing the cascaded filters with the direct path (as in \Figure{fig:phaser}) leads to the appearance of notches in the output spectrum of the phaser at frequencies for which the phase shift is an odd multiple of $-\pi$, i.e., $-\pi, -3\pi, -5\pi$, etc. Since each first-order allpass filter has a maximum negative phase shift of $-\pi$, $N$ allpass filters result in $\frac{N}{2}$ notches (assuming that $N$ is even).

The core idea of the phaser is to modulate the break frequencies of the allpass filters so that the notches sweep up and down the frequency range creating an audibly interesting effect. In analog phasers, which are implemented using analog electronic circuitry, the values of physical components are being modulated. In digital phasers, the implementers can decide whether to modulate the break frequencies directly or to modulate just the allpass coefficients \cite{Kiiski2016}. The difference between these two design decisions are shown in \Figure{fig:pink_noise_phasered}. Modulating the allpass coefficients in a fixed range yields similar notch variation curves for each break frequency (middle spectrogram). Modulating the break frequencies themselves using a traditional frequency modulation scheme yields larger frequency variations for high frequencies than for low frequencies.

\newcommand{\scaleboxsize}{0.6}
\newcommand{\subfigurewidththree}{0.3\textwidth}
\begin{figure}
    \begin{subfigure}{0.33\textwidth}
        \centering
        \scalebox{\scaleboxsize}{% This file was created by tikzplotlib v0.9.9.
\begin{tikzpicture}

\begin{axis}[
tick align=outside,
tick pos=left,
x grid style={white!69.0196078431373!black},
xlabel={Time [s]},
xmin=0, xmax=9.98458049886621,
xtick style={color=black},
y grid style={white!69.0196078431373!black},
ylabel={Frequency [kHz]},
ymin=0, ymax=22,
ytick style={color=black},
ytick={0,5,10,15,20,25},
yticklabels={0.0,2.5,5.0,7.5,10.0,12.5}
]
\addplot graphics [includegraphics cmd=\pgfimage,xmin=0, xmax=9.98458049886621, ymin=0, ymax=22] {figures/tikz/pink_noise_10s_stft-000.png};
\end{axis}

\end{tikzpicture}
}
    \end{subfigure}
    \begin{subfigure}{0.28\textwidth}
        \centering
        \scalebox{\scaleboxsize}{% This file was created by tikzplotlib v0.9.9.
\begin{tikzpicture}

\begin{axis}[
tick align=outside,
tick pos=left,
x grid style={white!69.0196078431373!black},
xlabel={Time [s]},
xmin=0, xmax=9.98458049886621,
xtick style={color=black},
y grid style={white!69.0196078431373!black},
ylabel={Frequency [kHz]},
ymin=0, ymax=22,
ytick style={color=black},
ytick={0,5,10,15,20,25},
yticklabels={0.0,2.5,5.0,7.5,10.0,12.5}
]
\addplot graphics [includegraphics cmd=\pgfimage,xmin=0, xmax=9.98458049886621, ymin=0, ymax=22] {pink_noise_10s-target_stft-000.png};
\end{axis}

\end{tikzpicture}
}
    \end{subfigure}
    \begin{subfigure}{0.34\textwidth}
        \centering
        \scalebox{0.61}{% This file was created by tikzplotlib v0.9.9.
\begin{tikzpicture}

\begin{axis}[
colorbar,
colorbar style={
%   title=Magnitude [dB],    
  ytick={-60,-40,-20,0,20},yticklabels={
  \(\displaystyle {\ensuremath{-}60}\),
  \(\displaystyle {\ensuremath{-}40}\),
  \(\displaystyle {\ensuremath{-}20}\),
  \(\displaystyle {0}\),
  \(\displaystyle {20}\)
},ylabel={}},
colormap/viridis,
point meta max=22.3262987886224,
point meta min=-66.7740552468175,
tick align=outside,
tick pos=left,
x grid style={white!69.0196078431373!black},
xlabel={Time [s]},
xmin=0, xmax=9.98458049886621,
xtick style={color=black},
y grid style={white!69.0196078431373!black},
ymajorticks=false,
% ylabel={Frequency [kHz]},
ymin=0, ymax=22,
ytick style={color=black},
ytick={0,5,10,15,20,25},
yticklabels={0.0,2.5,5.0,7.5,10.0,12.5}
]
\addplot graphics [includegraphics cmd=\pgfimage,xmin=0, xmax=9.98458049886621, ymin=0, ymax=22] {figures/tikz/pink_noise_10s_phasered_stft-000};
\end{axis}

\end{tikzpicture}
}
    \end{subfigure}
    \caption{The effect of a phaser with 10 allpass filters on a pink noise visualized using the spectrogram. (Left) Input signal. (Middle) Output of a phaser that has six out of ten allpass coefficients modulated. (Right) Output of a phaser that has the break frequencies of all of its allpass filters modulated.}
    \label{fig:pink_noise_phasered}
\end{figure}

%%%%%%%%%%%%%%%%%%%%%%%%%%%%%%%%%%%%%%%%%%%%%%%%%%%%%%%%%%%%%%%%%%%%%%%%%%%%%%%%%%%%%%%
\subsection{Phaser Modeling}
%%%%%%%%%%%%%%%%%%%%%%%%%%%%%%%%%%%%%%%%%%%%%%%%%%%%%%%%%%%%%%%%%%%%%%%%%%%%%%%%%%%%%%%

The phaser effect pedal was introduced in the 1960s \cite{PASPWEB2010} and quickly gained popularity in rock music \cite{Hartmann1978}. Especially popular were Univibe and MXR Phase 90 models, which currently can reach high prices on the second-hand market \cite{Eichas2014,Kiiski2016}. Therefore, a need arose to create faithful \ac{VA} models of famous phaser pedals.

The variety of approaches to phaser modeling is representative for any \ac{VA} model. Eichas et al. \cite{Eichas2014} analyze a circuit schematic of the MXR Phase 90 and derive a state-space model making heavy use of matrices. This is an example of a white-box approach.

Kiiski et al. \cite{Kiiski2016} use the chirp signal to measure the impulse response of the Fame Sweet Tone phaser, which in this approach is treated as a \ac{LTV} system. This is not a fully black-box approach because in the signal model design they make use of the circuit schematics to derive the modulation scheme. Thus, it falls under the category of grey-box modeling.

Wright and V{\"a}lim{\"a}ki \cite{Wright2020} provided a proof of concept that a phaser pedal can be modeled in an almost black-box fashion. The only deviation from the input-to-output relationship modeling is the estimation of the \ac{LFO} signal which is used to condition the neural network. The same type of data, i.e., the input signal and the (ground truth) \ac{LFO} signal, is used for this work.

This work differs from \cite{Wright2020} in that it examines a different approach to neural modeling of the phaser. Instead of using an \ac{RNN}, an \ac{MLP} is used to learn the derivative of the system with respect to time. The \ac{MLP} is conditioned on the input, the \ac{LFO} signal, and the output at any given time (not just at sample positions). A desirable property of a system capable of modeling the phaser in this way would be a smaller model size and computational complexity in comparison to an \ac{LSTM} network.
