%\section*{Abstract}
Virtual Analog modeling is the science of imitating analog audio systems by digital ones. It allows to obtain vintage sound of expensive or no longer manufactured equipment by means of software implementation. Analog systems, such as analog electronic circuits, are often described as state-space models governed by a system of ordinary differential equations. 

Traditionally, deep learning was used to approximate an input-to-output mapping. After the introduction of residual networks, which learn the residual of the mapping rather than the full mapping, the domains of neural networks and ordinary differential equations started to overlap. Recent developments suggest that neural networks are capable of learning the derivative of a dynamical system from data. The learned derivative may be supplied to a numerical solver of choice to compute the actual system output in response to an input signal. 

In this work, an application of the derivative-learning concept to Virtual Analog modeling is presented with a focus on two specific audio effects: a diode clipper (distortion) circuit and a phaser pedal. We show that the proposed framework is able to learn the derivative of the diode clipper. What is more, it improves its accuracy at sampling rates higher than the dataset sampling rate and for significantly higher sampling rates, it outperforms the baseline approach. For simple models, the proposed approach requires significantly fewer parameters than the state-of-the-art recurrent neural networks to obtain a comparable performance. Finally, the intricacies of learning an unknown derivative of a time-variant system are discussed.
