%\section*{Abstract}
Recent developments have shown that neural networks are capable of learning the derivative of a dynamical system from data. The learned derivative may be supplied to a numerical solver of choice to compute the actual system output in response to initial conditions. In this work, an application of the derivative-learning concept to Virtual Analog modeling is presented with a focus on two specific audio effects: a diode clipper (distortion) circuit and a phaser pedal. We propose how to include the input signal as an excitation of the system. We suggest how to handle the initial condition and how to augment the state vector of the system with latent states to improve the accuracy. We show that the proposed framework is able to learn the derivative of the diode clipper. What is more, it improves its accuracy at sampling rates higher than the sampling rate at training. For significantly higher sampling rates, it outperforms the baseline method. For simple models, the proposed approach requires significantly fewer parameters than the state-of-the-art recurrent neural networks to obtain a comparable performance. Finally, we discuss the intricacies of learning the unknown derivative of a time-variant system.
