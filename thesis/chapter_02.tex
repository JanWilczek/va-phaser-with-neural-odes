%%%%%%%%%%%%%%%%%%%%%%%%%%%%%%%%%%%%%%%%%%%%%%%%%%%%%%%%%%%%%%%%%%%%%%%%%%%%%%%%%%%%%%%
\cleardoublepage
\thispagestyle{empty}
\begin{center}
\vspace*{3cm}
{\huge \bf Part I}\\ \vspace*{1cm}
{\Huge \bf Title of this Part}\\\vspace*{0.2cm}
{\Huge \bf Being Longer than One Line}\\\vspace*{3cm}
\begin{figure}[ht]
\centering
\includegraphics[height=6cm]{figures/part1_notesAndWaveform_orange}
\end{figure}
\end{center}
\addcontentsline{toc}{part}{I\hspace {1em}Title of this Part Being Longer than One Line}
\label{par:part1}
\newpage
\quad
\thispagestyle{empty}
\newpage

%%%%%%%%%%%%%%%%%%%%%%%%%%%%%%%%%%%%%%%%%%%%%%%%%%%%%%%%%%%%%%%%%%%%%%%%%%%%%%%%%%%%%%%


%%%%%%%%%%%%%%%%%%%%%%%%%%%%%%%%%%%%%%%%%%%%%%%%%%%%%%%%%%%%%%%%%%%%%%%%%%%%%%%%%%%%%%%
\chapter{Theoretic Foundations}
\label{chapter:Foundations}
%%%%%%%%%%%%%%%%%%%%%%%%%%%%%%%%%%%%%%%%%%%%%%%%%%%%%%%%%%%%%%%%%%%%%%%%%%%%%%%%%%%%%%%

Each chapter should start with a small summary discussing the content and the relation
of the sections. In this chapter, we elaborate on the theoretical background, 
foundations and concepts that are being used in the thesis. In particular, we 
focus on latex construct that are typically used in thesis documents.
\Section{section:mathEquations} illustrates the usage of math equations.
Then, in \Section{section:citations}, some examples are given on how to cite literature.
In \Section{section:tables}, $\ldots$

When using labels, please avoid collisions in the label names.


%%%%%%%%%%%%%%%%%%%%%%%%%%%%%%%%%%%%%%%%%%%%%%%%%%%%%%%%%%%%%%%%%%%%%%%%%%%%%%%%%%%%%%%
\section{Math Equations}
\label{section:mathEquations}
%%%%%%%%%%%%%%%%%%%%%%%%%%%%%%%%%%%%%%%%%%%%%%%%%%%%%%%%%%%%%%%%%%%%%%%%%%%%%%%%%%%%%%%

You may want to display math equations in three distinct styles:
inline, numbered or non-numbered display. These three styles are used in the
following paragraph along with labeling, aligning and referencing equations.

A little anecdote tells that Johann Carl Friedrich Gau{\ss}\index{Gau{\ss}}
was able to compute the sum of numbers in the set
$\{i \in \N\,|\,i \leq N\}$ in a very short amount of time at the age of nine.
Apparently, he was able to observe that if $N$ is even, then
the numbers in the sum can be reordered and grouped so that
\begin{align}
	\sum_{i=1}^N i &= 1 + 2 + \ldots + N \notag \\
	               &= (1 + N) + (2 + N-1) + \ldots + (\frac{N}{2} + \frac{N}{2}+1) \label{eq:gaussSum}\\
	               &= \frac{N\cdot (N+1)}{2}                                            \label{eq:gaussFormula}
\end{align}
If $N$ is even, then the reordering Equation $\eqref{eq:gaussSum}$ simply builds $N/2$ summands
with value $N+1$, resulting to Equation $\eqref{eq:gaussFormula}$.
This Equation also holds for odd $N$ and a similar argument holds in this case.

%%%%%%%%%%%%%%%%%%%%%%%%%%%%%%%%%%%%%%%%%%%%%%%%%%%%%%%%%%%%%%%%%%%%%%%%%%%%%%%%%%%%%%%
\section{Citations}
\label{section:citations}
%%%%%%%%%%%%%%%%%%%%%%%%%%%%%%%%%%%%%%%%%%%%%%%%%%%%%%%%%%%%%%%%%%%%%%%%%%%%%%%%%%%%%%%

In a thesis one might want to cite interesting and relevant
books \cite{Mueller07_InformationRetrieval_SPRINGER,Habets07_SpeechDereverberation},
scientific papers \cite{HerreT11_AudioObjects_AES,EdlerD09_TimeWarpedDCT},
or websites \cite{Mutopia_website}.

%%%%%%%%%%%%%%%%%%%%%%%%%%%%%%%%%%%%%%%%%%%%%%%%%%%%%%%%%%%%%%%%%%%%%%%%%%%%%%%%%%%%%%%
\section{Tables}
\label{section:tables}
%%%%%%%%%%%%%%%%%%%%%%%%%%%%%%%%%%%%%%%%%%%%%%%%%%%%%%%%%%%%%%%%%%%%%%%%%%%%%%%%%%%%%%%

Because tables cannot, (well, at least not easily) be split across pages, the best
placement for them is typically the top of the page
nearest their initial cite.  To
ensure this proper ``floating'' placement of tables, use the
environment \emph{table} to enclose the table's contents and
the table caption.  The contents of the table itself must go
in the \emph{tabular} environment, to
be aligned properly in rows and columns, with the desired
horizontal and vertical rules, as seen in \Table{table:characters}.

\begin{table}[t]
\centering
\renewcommand{\arraystretch}{1.3}
\begin{tabular}{@{}cll@{}} 
\toprule
Non-English or Math & Frequency & Comments\\ 
\midrule
\O & 1 in 1,000& For Swedish names\\
$\pi$ & 1 in 5& Common in math\\
\$ & 4 in 5 & Used in business\\
$\Psi^2_1$ & 1 in 40,000& Unexplained usage\\
\bottomrule
\end{tabular}
\caption{Frequency of Special Characters. Note that this table does not contain any
vertical lines which makes the table look more tidy.}
\label{table:characters}
\end{table}

%%%%%%%%%%%%%%%%%%%%%%%%%%%%%%%%%%%%%%%%%%%%%%%%%%%%%%%%%%%%%%%%%%%%%%%%%%%%%%%%%%%%%%%
\section{Figures}
\label{section:figures}
%%%%%%%%%%%%%%%%%%%%%%%%%%%%%%%%%%%%%%%%%%%%%%%%%%%%%%%%%%%%%%%%%%%%%%%%%%%%%%%%%%%%%%%

Like tables, figures cannot be split across pages; the
best placement for them
is typically the top or the bottom of the page nearest
their initial cite.  To ensure this proper ``floating'' placement
of figures, use the environment
\emph{figure} to enclose the figure and its caption.

This sample thesis document contains examples of \emph{.pdf}
files to be displayable with \LaTeX.

A simple figure can be created as shown in \Figure{figure:sine}.
A figure with multiple subfigures can be seen in \Figure{figure:sines}.
The subfigures can be referenced as \Figure{figure:sine_lf} and \Figure{figure:sine_hf}.

\begin{figure}[t]
  \centering
  \includegraphics[width=0.5\textwidth]{figures/fig_sin1.pdf}
  \caption{Sine function.}
  \label{figure:sine}
\end{figure}


\begin{figure}[t]
    \centering
    \begin{subfigure}[b]{0.45\textwidth}
        \includegraphics[width=\textwidth]{figures/fig_sin1.pdf}
        \caption{Low frequency sine}
        \label{figure:sine_lf}
    \end{subfigure}
    \begin{subfigure}[b]{0.45\textwidth}
        \includegraphics[width=\textwidth]{figures/fig_sin2.pdf}
        \caption{High frequency sine}
        \label{figure:sine_hf}
    \end{subfigure}
    \caption{Sine functions with different frequencies.}
    \label{figure:sines}
\end{figure}


%%%%%%%%%%%%%%%%%%%%%%%%%%%%%%%%%%%%%%%%%%%%%%%%%%%%%%%%%%%%%%%%%%%%%%%%%%%%%%%%%%%%%%%
\section{Theorem-like Constructs}
\label{section:constructs}
%%%%%%%%%%%%%%%%%%%%%%%%%%%%%%%%%%%%%%%%%%%%%%%%%%%%%%%%%%%%%%%%%%%%%%%%%%%%%%%%%%%%%%%
Other common constructs that may occur in your thesis are
the forms for logical constructs like theorems, axioms,
corollaries and proofs. See the following example:

\begin{Theorem}
Let $f$ be continuous on $[a,b]$.  If $G$ is
an antiderivative for $f$ on $[a,b]$, then
\begin{align}
  \int^b_af(t)dt = G(b) - G(a).
\end{align}
\end{Theorem}

The following is a definition:
\begin{Definition}
If $z$ is irrational, then by $e^z$ we mean the
unique number which has logarithm $z$:
\begin{align}
  \log e^z = z.
\end{align}
\end{Definition}


